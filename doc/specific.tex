\Chapter{Specific methods}

This chapter describes methods of {\LOOPS} that apply to some special loops,
mostly Bol and Moufang loops.

%%%%%%%%%%%%%%%%%%%%%%%%%%%%%%%%%%%%%%%%%%%%%%%%%%%%%%%%%%%%%%%%%%%%%%%%%%%%%%%
\Section{Core methods for Bol loops}

Let $L$ be a left Bol loop\index{left Bol loop} such that the mapping $x\mapsto
x^2$ is a permutation of $L$. Define a new operation $*$ on $L$ by $x*y
=(x(y^2x))^{1/2}$. Then Bruck showed that $(L,*)$ is a left Bruck loop, called
the <assoicated left Bruck loop>.\index{associated left Bruck loop}. (In fact,
Bruck used the isomorphic operation $x*y = x^{1/2}(yx^{1/2})$ instead. Our
approach is more natural, since the associated left Bruck loop to a left Bruck
loop $L$ is identical to $L$ then.) The associated left Bruck loop of $L$ is
returned via

\>AssoicatedLeftBruckLoop( <L> ) A


%%%%%%%%%%%%%%%%%%%%%%%%%%%%%%%%%%%%%%%%%%%%%%%%%%%%%%%%%%%%%%%%%%%%%%%%%%%%%%%
\Section{Moufang modifications}

Ale\v{s} Dr\'apal discovered two prominent families of extensions of Moufang
loops. It turns out that these extensions can be used to obtain all
nonassociative Moufang loops of order at most $64$. We call these two
constructions <Moufang modifications>\index{Moufang modifications}. The library
of Moufang loops included with {\LOOPS} is based on Moufang modifications. We
describe the two modifications briefly here. See \cite{DrVo} for details.

Assume that $L$ is a Moufang loop with normal subloop $S$ such that $L/S$ is a
cyclic group of order $2m$. Let $h\in S\cap Z(L)$. Let $\alpha$ be a generator
of $L/S$ and write $L = \bigcup_{i\in M} \alpha^i$, where $M=\{-m+1$, $\dots$,
$m\}$. Let $\sigma:\Z\to M$ be defined by $\sigma(i)=0$ if $i\in M$,
$\sigma(i)=1$ if $i>m$, and $\sigma(i)=-1$ if $i\<-m+1$. Introduce a new
multiplication $\ast$ on $L$ defined by
$$
    x\ast y = xyh^{\sigma(i+j)},
$$
where $x\in \alpha^i$, $y\in\alpha^j$, $i\in M$, $j\in M$. Then $(L,\ast)$ is a
Moufang loop, a <cyclic modification>\index{cyclic modification} of $L$.

When $L$, $S$, $\alpha$, $h$ are as above and when $a$ is any element of
$\alpha$, the corresponding cyclic modification is obtained via

\>LoopByCyclicModification( <L>, <S>, <a>, <h> ) F

Now assume that $L$ is a Moufang loop with normal subloop $S$ such that $L/S$
is a dihedral group of order $4m$, with $m\ge 1$. Let $M$ and $\sigma$ be
defined as in the cyclic case. Let $\beta$, $\gamma\in L/S$ be two involutions
of $L/S$ such that $\alpha=\beta\gamma$ generates a cyclic subgroup of $L/S$ of
order $2m$. Let $e\in\beta$ and $f\in\gamma$ be arbitrary. Then $L$ can be
written as a disjoint union $L=\bigcup_{i\in M}(\alpha^i\cup e\alpha^i)$, and
also $L=\bigcup_{i\in M}(\alpha^i\cup \alpha^if$. Let $G_0=\bigcup_{i\in
M}\alpha^i$, and $G_1=L\setminus G_0$. Let $h\in S\cap N(L)\cap Z(G_0)$.
Introduce a new multiplication $\ast$ on $L$ defined by
$$
    x\ast y = xyh^{(-1)^r\sigma(i+j)},
$$
where $x\in\alpha^i\cup e\alpha^i$, $y\in\alpha^j\cup \alpha^jf$, $i\in M$,
$j\in M$, $y\in G_r$, $r\in\{0,1\}$. Then $(L,\ast)$ is a Moufang loop, a
<dihedral modification>\index{dihedral modification} of $L$.

When $L$, $S$, $e$, $f$ and $h$ are as above, the corresponding dihedral
modification is obtained via

\>LoopByDihedralModification( <L>, <S>, <e>, <f>, <h> ) F

In order to apply the cyclic and dihedral modifications, it is beneficial to
have access to a class of nonassociative Moufang loops. The following
construction is due to Chein:

Let $G$ be a group. Let $\overline{G}=\{\overline{g};\;g\in G\}$ be a set of
new elements. Define multiplication $\ast$ on $L=G\cup \overline{G}$ by
$$
    g\ast h = gh,\ g\ast\overline{h}=\overline{hg},\
    \overline{g}\ast h = \overline{gh^{-1}},\
    \overline{g}\ast \overline{h}=h^{-1}g,
$$
where $g$, $h\in G$. Then $L=M(G,2)$ is a Moufang loop that is nonassociative
if and only if $G$ is nonabelian.

The loop $M(G,2)$ can be obtained from a finite group $G$ with

\>LoopMG2( <G> ) F

%%%%%%%%%%%%%%%%%%%%%%%%%%%%%%%%%%%%%%%%%%%%%%%%%%%%%%%%%%%%%%%%%%%%%%%%%%%%%%%
\Section{Triality for Moufang loops}

Let $G$ be a group and $\sigma$, $\rho$ be automorphisms of $G$, satisfying
$\sigma^2 = \rho^3 = (\sigma \rho)^2 = 1$. We write the automorphisms of a
group as exponents and $[g,\sigma]$ for $g^{-1}g^\sigma$. We say that the
triple $(G,\rho,\sigma)$ is a <group with triality>\index{group with triality}
if $[g, \sigma] [g,\sigma]^\rho [g,\sigma]^{\rho^2} =1$ holds for all $g \in
G$. It is known that one can associate a group with triality $(G,\rho,\sigma)$
in a canonical way with a Moufang loop $L$. See \cite{NaVo2003} for more
details.

For any Moufang loop $L$, we can calculate the triality group as a permutation
group acting on $3|L|$ points. If the multiplication group of $L$ is
polycyclic, then we can also represent the triality group as a pc group. In
both cases, the automorphisms $\sigma$ and $\rho$ are in the same family as the
elements of $G$.

Given a Moufang loop $L$, the function

\>TrialityPermGroup( <L> ) F

returns a record [$G$, $\rho$, $\sigma$], where $G$ is the group with triality
associated with $L$, and $\rho$, $\sigma$ are the corresponding triality
automorphisms.

The function

\>TrialityPcGroup( <L> ) F

differs from `TrialityPermGroup' only in that $G$ is returned as a pc group.
