\Chapter{Introduction}

{\LOOPS} is a package for \package{GAP4} whose purpose is to:
\beginlist%unordered
\item{$\circ$}
    provide researchers in nonassociative algebra with a powerful computational tool
    concerning finite loops and quasigroups,
\item{$\circ$}
    extend {\GAP} toward the realm of nonassociative structures.
\endlist

%%%%%%%%%%%%%%%%%%%%%%%%%%%%%%%%%%%%%%%%%%%%%%%%%%%%%%%%%%%%%%%%%%%%%%%%%%%%%%%
\Section{Installation}

We assume that you have \package{GAP 4.4} or newer installed on your computer.
Download the {\LOOPS} package from the distribution website

\URL{http://www.math.du.edu/loops}

and unpack the downloaded file into the `pkg' subfolder of your {\GAP} folder.

After this step, there should be a subfolder `loops' in your `pkg' folder. The
package {\LOOPS} can then be loaded to {\GAP} anytime by calling
\begintt
LoadPackage("loops");
\endtt
If you wish to load {\LOOPS} automatically while starting {\GAP}, open the file
`loops/PackageInfo.g', and change `Autoload:=false' to
`Autoload:=true' in the file.

%%%%%%%%%%%%%%%%%%%%%%%%%%%%%%%%%%%%%%%%%%%%%%%%%%%%%%%%%%%%%%%%%%%%%%%%%%%%%%%
\Section{Documentation}

The documentation is available is several formats: \TeX, pdf, dvi, ps, html,
and as an online help in {\GAP}. All these formats have been obtained directly
from the master {\TeX} documentation file. Consequently, the different formats
of the documentation differ only in their appearance, not in content.

All formats of the documentation except html can be found in the `doc' folder
of {\LOOPS}, while the html version is in the `htm' folder. You can also
download the documentation at the {\LOOPS} distribution website.

The online {\GAP} help is available upon installing {\LOOPS}, and can be
accessed in the usual way, i.e., upon typing `?'<command>, {\GAP} displays the
section of the {\LOOPS} manual containing information about <command>.

%%%%%%%%%%%%%%%%%%%%%%%%%%%%%%%%%%%%%%%%%%%%%%%%%%%%%%%%%%%%%%%%%%%%%%%%%%%%%%%
\Section{Test files}

Test files conforming to the {\GAP} standards are provided for {\LOOPS}. They
can be found in the folder `tst'. The command `ReadPackage("loops",
"tst/testall.g")' runs all tests for {\LOOPS}.

%%%%%%%%%%%%%%%%%%%%%%%%%%%%%%%%%%%%%%%%%%%%%%%%%%%%%%%%%%%%%%%%%%%%%%%%%%%%%%%
\Section{Feedback}

We welcome all comments and suggestions on {\LOOPS}, especially those
concerning the future development of the package. You can contact us by e-mail.

%%%%%%%%%%%%%%%%%%%%%%%%%%%%%%%%%%%%%%%%%%%%%%%%%%%%%%%%%%%%%%%%%%%%%%%%%%%%%%%
\Section{Acknowledgement}

We thank the following people for sending us remarks and comments, and for
suggesting new functionality of the package: Muniru Asiru, Bjoern Assmann,
Andreas Distler, Steve Flammia, Kenneth W. Johnson, Michael K. Kinyon, Frank
L\accent127ubeck, and Jonathan D. H. Smith.

G\'abor P. Nagy was supported by OTKA grants F042959 and T043758, and Petr
Vojt\v{e}chovsk\'y by the 2006 PROF grant of the University of Denver.
